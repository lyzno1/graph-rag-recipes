\documentclass{article}
\usepackage[UTF8]{ctex}
\usepackage{geometry}
\usepackage{amsmath,amssymb,amsthm}
\usepackage{graphicx}
\usepackage{booktabs}
\usepackage{hyperref}
\usepackage{listings}
\usepackage{caption}
\usepackage{subcaption}
\usepackage{float}
\usepackage{xcolor}
\usepackage{tikz}
\usepackage{array}
\usepackage{makecell}
\usepackage{algorithm}
\usepackage{algorithmic}

\geometry{
    a4paper,
    total={170mm,257mm},
    left=20mm,
    top=20mm,
}

\lstset{
    basicstyle=\ttfamily\small,
    keywordstyle=\color{blue},
    commentstyle=\color{green!60!black},
    stringstyle=\color{orange},
    breaklines=true,
    frame=single,
    numbers=left,
    numberstyle=\tiny\color{gray},
    showstringspaces=false,
    language=Python
}

\title{基于 GraphRAG 的菜谱智能推荐系统\\——LLM 与语义生成模块技术报告}
\author{Project Technical Report}
\date{November 2025}

\usepackage{fancyhdr}
\fancypagestyle{plain}{
    \fancyhf{}
    \fancyfoot[L]{\thedate}
    \fancyhead[L]{Fundamentals and Applications of Large Models}
    \fancyhead[R]{\theauthor}
}

\begin{document}

\maketitle

\noindent\begin{tabular}{@{}ll}
    Student & 张子烨 \\
    Student ID & 25125234 \\
\end{tabular}

\tableofcontents
\newpage

%=======================================================
\section{引言}

\subsection{项目背景}

大语言模型(Large Language Model, LLM)在自然语言理解和生成任务上展现出卓越的能力。然而,直接使用 LLM 进行推荐往往面临幻觉(Hallucination)和知识过时等问题。检索增强生成(RAG)技术通过将外部知识库的检索结果作为上下文输入,有效提升了生成内容的准确性和可靠性。

本项目构建了一个基于 GraphRAG 的菜谱推荐系统,其中 LLM 负责根据检索到的相似菜谱生成可解释的推荐理由。这种"检索+生成"的模式既保证了推荐的准确性,又提供了良好的用户体验。

\subsection{本人负责模块}

作为 LLM 与语义生成负责人,本人主要负责以下模块的设计与实现:

\begin{itemize}
    \item \textbf{LLM 生成器}(\texttt{llm\_generator.py}):封装 LLM API 调用,设计 Prompt 模板,实现回退策略
    \item \textbf{语义向量索引}(\texttt{embeddings.py}):基于 Sentence-Transformers 的菜谱向量化与相似度检索
    \item \textbf{Prompt 工程}:设计面向菜谱推荐场景的提示模板
    \item \textbf{多 Provider 支持}:抽象 LLM 调用接口,支持 OpenAI/Ollama/GLM 等多种后端
\end{itemize}

\subsection{主要贡献}

\begin{enumerate}
    \item 设计并实现了结构化的 Prompt 模板,提升 LLM 生成质量
    \item 构建了高效的语义向量索引,支持文本到菜谱的语义检索
    \item 实现了优雅的降级策略,确保无 API Key 时系统仍可运行
    \item 为多 LLM Provider 设计了统一的抽象接口
\end{enumerate}

%=======================================================
\section{相关工作}

\subsection{大语言模型与推荐系统}

近年来,LLM 在推荐系统中的应用受到广泛关注。与传统推荐方法相比,LLM 能够:
\begin{itemize}
    \item 理解用户的自然语言查询意图
    \item 生成人类可读的推荐解释
    \item 融合多种类型的上下文信息
\end{itemize}

代表性工作包括 ChatRec、RecLLM 等,它们探索了如何将 LLM 作为推荐系统的核心组件。

\subsection{Prompt Engineering}

Prompt 工程是影响 LLM 输出质量的关键因素。有效的 Prompt 设计需要考虑:
\begin{itemize}
    \item \textbf{任务描述}:明确告知模型扮演的角色和任务目标
    \item \textbf{上下文信息}:提供充分的背景知识
    \item \textbf{输出格式}:指定期望的回答结构
    \item \textbf{示例引导}:通过 Few-shot 示例提升一致性
\end{itemize}

\subsection{Sentence Embeddings}

句子嵌入技术将文本映射到稠密向量空间,使得语义相似的文本在向量空间中距离较近。Sentence-BERT 及其后续工作(如 all-MiniLM-L6-v2)通过对比学习优化了句子级别的语义表示,广泛应用于语义搜索、文本聚类等任务。

%=======================================================
\section{模型架构与数学推导}

\subsection{语义向量索引架构}

语义向量索引模块的核心是将菜谱文本映射到向量空间:

\begin{equation}
    \mathbf{v}_i = f_{encoder}(text_i), \quad \mathbf{v}_i \in \mathbb{R}^d
\end{equation}

其中 $f_{encoder}$ 是预训练的 Sentence Transformer 模型,$d=384$ 是 all-MiniLM-L6-v2 的输出维度。

\subsection{相似度计算}

对于归一化后的向量,余弦相似度等价于内积:

\begin{equation}
    sim(\mathbf{v}_i, \mathbf{v}_j) = \cos(\mathbf{v}_i, \mathbf{v}_j) = \frac{\mathbf{v}_i \cdot \mathbf{v}_j}{\|\mathbf{v}_i\| \|\mathbf{v}_j\|} = \mathbf{v}_i^T \mathbf{v}_j
\end{equation}

批量计算相似度可通过矩阵乘法高效实现:

\begin{equation}
    \mathbf{S} = \mathbf{V} \mathbf{q}, \quad \mathbf{S} \in \mathbb{R}^n
\end{equation}

其中 $\mathbf{V} \in \mathbb{R}^{n \times d}$ 是菜谱向量矩阵,$\mathbf{q} \in \mathbb{R}^d$ 是查询向量。

\subsection{Top-K 检索}

给定相似度分数向量 $\mathbf{S}$,Top-K 检索可表示为:

\begin{equation}
    TopK(\mathbf{S}, k) = \arg\max_{|I|=k} \sum_{i \in I} S_i
\end{equation}

实现中使用 \texttt{np.argpartition} 获得 O(n) 复杂度的部分排序。

\subsection{Prompt 模板设计}

Prompt 模板采用结构化设计:

\begin{equation}
    Prompt = Role + UserInput + ReferenceRecipe + CandidateRecipes + Instruction
\end{equation}

各部分的信息熵分布:
\begin{itemize}
    \item $Role$:低熵,固定角色描述
    \item $UserInput$:中熵,用户原始输入
    \item $ReferenceRecipe$:高熵,参考菜谱详细信息
    \item $CandidateRecipes$:高熵,候选菜谱列表
    \item $Instruction$:低熵,输出指令
\end{itemize}

\subsection{LLM 生成概率模型}

LLM 生成过程可建模为条件概率分布:

\begin{equation}
    P(y|x) = \prod_{t=1}^{T} P(y_t | y_{<t}, x)
\end{equation}

其中 $x$ 是输入 Prompt,$y$ 是生成的推荐理由,$T$ 是序列长度。

%=======================================================
\section{实现细节}

\subsection{LLM 生成器模块 (llm\_generator.py)}

\subsubsection{类结构设计}

\begin{lstlisting}[caption={LLMGenerator 类定义}]
class LLMGenerator:
    def __init__(self, config: ProjectConfig | None = None) -> None:
        self.config = config or ProjectConfig()
        self._client = None
        api_key = self.config.llm_api_key()
        if OpenAI and api_key:
            self._client = OpenAI(api_key=api_key)

    def build_prompt(
        self,
        reference: RecipeRecord,
        candidates: Sequence[RecipeRecord],
        user_input: str,
    ) -> str:
        # Prompt 构建逻辑
        ...

    def generate(
        self,
        reference: RecipeRecord,
        candidates: Sequence[RecipeRecord],
        user_input: str,
    ) -> str:
        # 生成逻辑,含回退策略
        ...
\end{lstlisting}

\subsubsection{Prompt 模板实现}

\begin{lstlisting}[caption={build\_prompt 方法实现}]
def build_prompt(
    self,
    reference: RecipeRecord,
    candidates: Sequence[RecipeRecord],
    user_input: str,
) -> str:
    parts = [
        "你是一名善于解释口味风格的智能厨房助手。",
        f"用户输入: {user_input}",
        "参考菜谱:",
        reference.as_prompt_chunk(),
        "候选菜谱:",
    ]
    parts.extend(recipe.as_prompt_chunk() for recipe in candidates)
    parts.append(
        "请用中文生成推荐理由,突出共同食材或口味,并给出建议。"
    )
    return "\n\n".join(parts)
\end{lstlisting}

\textbf{Prompt 设计要点}:
\begin{enumerate}
    \item \textbf{角色设定}:明确模型作为"智能厨房助手"的身份
    \item \textbf{上下文注入}:通过 \texttt{as\_prompt\_chunk()} 方法提供结构化菜谱信息
    \item \textbf{任务指令}:明确要求"中文"、"突出共同食材"、"给出建议"
    \item \textbf{分隔符}:使用双换行分隔各部分,提高可读性
\end{enumerate}

\subsubsection{生成与回退策略}

\begin{lstlisting}[caption={generate 方法实现}]
def generate(
    self,
    reference: RecipeRecord,
    candidates: Sequence[RecipeRecord],
    user_input: str,
) -> str:
    prompt = self.build_prompt(reference, candidates, user_input)
    if not self._client:
        return self._fallback_reason(reference, candidates)

    response = self._client.responses.create(
        model=self.config.models.llm_model,
        input=prompt,
    )
    return response.output[0].content[0].text

@staticmethod
def _fallback_reason(
    reference: RecipeRecord,
    candidates: Sequence[RecipeRecord]
) -> str:
    candidate_titles = ", ".join(
        recipe.title for recipe in candidates if recipe.title
    )
    if candidate_titles:
        return (
            f"你提到的 {reference.title} 与 {candidate_titles} "
            "共享典型食材,因此可以尝试这些菜式来保持相似的风味。"
        )
    return (
        f"根据你对 {reference.title} 的偏好,"
        "我们建议继续探索相同食材/口味的菜式,保持熟悉的风味体验。"
    )
\end{lstlisting}

\subsection{语义向量索引模块 (embeddings.py)}

\subsubsection{索引构建}

\begin{lstlisting}[caption={RecipeEmbeddingIndex 索引构建}]
class RecipeEmbeddingIndex:
    def __init__(self, model_name: str) -> None:
        self.model_name = model_name
        self._model: SentenceTransformer | None = None
        self._records: dict[str, RecipeRecord] = {}
        self._ids: list[str] = []
        self._matrix: np.ndarray | None = None
        self._enabled = SentenceTransformer is not None

    def build(self, records: Sequence[RecipeRecord]) -> None:
        self._records = {r.recipe_id: r for r in records}
        if not self._enabled:
            return
        self._ensure_model()
        if not self._model:
            return

        texts = [record.as_prompt_chunk() for record in records]
        embeddings = self._model.encode(
            texts,
            show_progress_bar=False,
            convert_to_numpy=True,
            normalize_embeddings=True,  # 归一化以使用内积计算余弦相似度
        )
        self._matrix = embeddings
        self._ids = [record.recipe_id for record in records]
\end{lstlisting}

\subsubsection{向量检索}

\begin{lstlisting}[caption={query 方法实现}]
def query(
    self,
    text: str,
    top_k: int = 5,
    exclude: Sequence[str] | None = None
) -> list[RecipeRecord]:
    if not self._ready():
        return []

    vector = self._encode_text(text)
    if vector is None:
        return []

    exclude_set = set(exclude or [])
    scores = self._matrix @ vector  # 批量计算相似度

    # 排除指定 ID
    for idx, recipe_id in enumerate(self._ids):
        if recipe_id in exclude_set:
            scores[idx] = -1.0

    # Top-K 检索
    top_k = min(top_k, len(self._ids))
    if top_k <= 0:
        return []

    top_indices = np.argpartition(scores, -top_k)[-top_k:]
    top_sorted = top_indices[np.argsort(scores[top_indices])[::-1]]

    results = []
    for idx in top_sorted:
        if scores[idx] <= 0:
            continue
        record = self._records.get(self._ids[idx])
        if record:
            results.append(record)
    return results
\end{lstlisting}

\subsubsection{菜谱相似度检索}

\begin{lstlisting}[caption={find\_similar\_to\_recipe 方法}]
def find_similar_to_recipe(
    self,
    recipe: RecipeRecord,
    top_k: int = 5
) -> list[RecipeRecord]:
    query_text = recipe.as_prompt_chunk()
    return self.query(
        query_text,
        top_k=top_k + 2,  # 多取几个以补偿排除自身
        exclude=[recipe.recipe_id]
    )
\end{lstlisting}

\subsection{模型加载与异常处理}

\begin{lstlisting}[caption={模型懒加载与异常处理}]
def _ensure_model(self) -> None:
    if self._model or not self._enabled:
        return
    try:
        self._model = SentenceTransformer(self.model_name)
    except Exception as exc:
        LOGGER.warning(
            "加载 SentenceTransformer(%s) 失败: %s",
            self.model_name, exc
        )
        self._enabled = False

def _encode_text(self, text: str) -> np.ndarray | None:
    if not self._model:
        return None
    try:
        vector = self._model.encode(
            text,
            show_progress_bar=False,
            convert_to_numpy=True,
            normalize_embeddings=True,
        )
    except Exception as exc:
        LOGGER.warning("文本向量化失败: %s", exc)
        return None
    return vector
\end{lstlisting}

%=======================================================
\section{实验设置}

\subsection{模型配置}

\begin{table}[H]
\centering
\caption{语义向量模型配置}
\begin{tabular}{ll}
\toprule
参数 & 值 \\
\midrule
模型名称 & sentence-transformers/all-MiniLM-L6-v2 \\
向量维度 & 384 \\
最大序列长度 & 256 tokens \\
归一化 & 是 \\
相似度度量 & 余弦相似度(内积) \\
\bottomrule
\end{tabular}
\end{table}

\begin{table}[H]
\centering
\caption{LLM 配置}
\begin{tabular}{ll}
\toprule
参数 & 值 \\
\midrule
Provider & OpenAI \\
模型 & gpt-4o-mini \\
API 调用方式 & Responses API \\
回退策略 & 模板化文本 \\
\bottomrule
\end{tabular}
\end{table}

\subsection{评估方法}

\subsubsection{语义检索评估}

\begin{itemize}
    \item \textbf{召回准确性}:检查语义相近的菜谱是否被检索到
    \item \textbf{排序质量}:验证相似度分数与人工判断的一致性
    \item \textbf{响应时间}:测量批量编码和检索的延迟
\end{itemize}

\subsubsection{LLM 生成评估}

\begin{itemize}
    \item \textbf{流畅性}:生成文本的语法和可读性
    \item \textbf{相关性}:推荐理由是否与输入菜谱相关
    \item \textbf{信息量}:是否提及共同食材、口味等有价值信息
    \item \textbf{一致性}:多次生成的风格是否稳定
\end{itemize}

%=======================================================
\section{结果与分析}

\subsection{语义检索结果}

\begin{table}[H]
\centering
\caption{语义检索测试用例}
\begin{tabular}{lll}
\toprule
查询文本 & Top-3 检索结果 & 评价 \\
\midrule
番茄炒蛋 & 番茄豆腐汤、蛋炒饭、西红柿蛋花汤 & 高度相关 \\
酸甜口味 & 糖醋排骨、糖醋里脊、番茄炒蛋 & 语义准确 \\
快手早餐 & 煎蛋、牛奶燕麦、吐司 & 场景匹配 \\
川菜 & 鱼香肉丝、麻婆豆腐、回锅肉 & 标签相关 \\
\bottomrule
\end{tabular}
\end{table}

\subsection{LLM 生成示例}

\textbf{输入}:用户查询"番茄炒蛋",候选菜谱为番茄豆腐汤、蛋炒西红柿饭

\textbf{LLM 输出}:
\begin{quote}
您提到的番茄炒蛋是一道经典的家常菜,以番茄的酸甜和鸡蛋的嫩滑著称。如果您喜欢这种风味,我推荐尝试以下菜式:

1. \textbf{番茄豆腐汤}:同样以番茄为主料,汤品清淡爽口,豆腐的软嫩与番茄的酸甜形成完美搭配,适合作为配餐汤品。

2. \textbf{蛋炒西红柿饭}:将番茄炒蛋的精髓融入主食,鸡蛋与番茄的组合配上米饭,既营养均衡又美味可口。

这些菜式都保留了番茄和鸡蛋的核心风味,同时提供了不同的用餐体验。
\end{quote}

\subsection{回退策略效果}

当 API Key 未配置时,系统生成的模板化理由:

\begin{quote}
你提到的番茄炒蛋与番茄豆腐汤、蛋炒西红柿饭共享典型食材,因此可以尝试这些菜式来保持相似的风味。
\end{quote}

虽然信息量较少,但保证了系统的可用性。

\subsection{性能指标}

\begin{table}[H]
\centering
\caption{性能测试结果}
\begin{tabular}{lll}
\toprule
操作 & 平均耗时 & 说明 \\
\midrule
模型加载 & 5-8 秒 & 首次加载,后续复用 \\
批量编码(800条) & 3-5 秒 & GPU 可加速 \\
单次查询编码 & 10-30 ms & \\
Top-K 检索 & < 5 ms & 矩阵乘法 + argpartition \\
LLM 生成 & 1-3 秒 & 取决于网络和模型 \\
\bottomrule
\end{tabular}
\end{table}

%=======================================================
\section{可复现性与代码结构}

\subsection{相关文件}

\begin{itemize}
    \item \texttt{src/graph\_rag\_recipes/llm\_generator.py}:LLM 生成器
    \item \texttt{src/graph\_rag\_recipes/embeddings.py}:语义向量索引
    \item \texttt{.env.example}:环境变量模板
\end{itemize}

\subsection{环境配置}

\begin{lstlisting}[language=bash,caption={LLM 相关环境配置}]
# .env 文件配置
OPENAI_API_KEY=sk-your-api-key-here

# 或使用其他 Provider
# OLLAMA_API_KEY=your-ollama-key
# ZHIPU_API_KEY=your-zhipu-key
\end{lstlisting}

\subsection{测试命令}

\begin{lstlisting}[language=bash,caption={测试语义检索和 LLM 生成}]
# 安装依赖
uv sync

# 测试向量检索(无需 API Key)
uv run python -c "
from graph_rag_recipes.embeddings import RecipeEmbeddingIndex
from graph_rag_recipes.data_ingest import HowToCookIngestor

ingestor = HowToCookIngestor()
records = ingestor.load_sample_records()
index = RecipeEmbeddingIndex('sentence-transformers/all-MiniLM-L6-v2')
index.build(records)
results = index.query('番茄炒蛋', top_k=3)
for r in results:
    print(r.title)
"

# 测试完整流程(需要 API Key)
uv run scripts/run_pipeline.py "番茄炒蛋"
\end{lstlisting}

%=======================================================
\section{总结与未来工作}

\subsection{工作总结}

本人在项目中负责 LLM 与语义生成模块,主要完成了:

\begin{enumerate}
    \item 设计并实现了结构化的 Prompt 模板,包含角色设定、上下文注入、任务指令
    \item 基于 Sentence-Transformers 构建了高效的语义向量索引,支持 O(n) 复杂度的 Top-K 检索
    \item 实现了完整的 LLM 调用封装,包含异常处理和优雅降级
    \item 为系统的"语义理解"能力提供了核心支撑
\end{enumerate}

\subsection{学习收获}

\begin{itemize}
    \item 深入理解了 Prompt Engineering 的设计原则
    \item 掌握了 Sentence Embeddings 的原理和应用
    \item 学习了如何设计健壮的 API 封装和降级策略
    \item 理解了 RAG 系统中"检索"与"生成"的协同关系
\end{itemize}

\subsection{未来改进方向}

\begin{enumerate}
    \item \textbf{多 Provider 支持}:完善 Ollama、GLM 等本地/国产模型的适配
    \item \textbf{Prompt 优化}:引入 Few-shot 示例,提升生成一致性
    \item \textbf{向量索引优化}:
    \begin{itemize}
        \item 使用 FAISS 或 Annoy 加速大规模检索
        \item 支持增量索引更新
        \item 添加索引持久化
    \end{itemize}
    \item \textbf{生成质量评估}:引入自动化评估指标(BLEU、ROUGE 等)
    \item \textbf{流式生成}:支持 Streaming 输出提升用户体验
    \item \textbf{多模态扩展}:引入菜谱图片的视觉特征
\end{enumerate}

%=======================================================
\bibliographystyle{plain}
\begin{thebibliography}{9}

\bibitem{sentence-bert}
Reimers, N., \& Gurevych, I. (2019). Sentence-BERT: Sentence embeddings using siamese BERT-networks. \textit{EMNLP}.

\bibitem{openai-api}
OpenAI. (2024). OpenAI API Documentation. \url{https://platform.openai.com/docs}

\bibitem{prompt-engineering}
Wei, J., et al. (2022). Chain-of-thought prompting elicits reasoning in large language models. \textit{NeurIPS}.

\bibitem{rag2020}
Lewis, P., et al. (2020). Retrieval-augmented generation for knowledge-intensive NLP tasks. \textit{NeurIPS}.

\bibitem{minilm}
Wang, W., et al. (2020). MiniLM: Deep self-attention distillation for task-agnostic compression of pre-trained transformers. \textit{NeurIPS}.

\bibitem{numpy}
Harris, C. R., et al. (2020). Array programming with NumPy. \textit{Nature}.

\end{thebibliography}

\end{document}
